\documentclass[10pt,letterpaper]{article}
\usepackage[latin1]{inputenc}
\usepackage{amsmath}
\usepackage{amsfonts}
\usepackage{amssymb}
\usepackage{graphicx}
% \usepackage[left=0.50in, right=0.50in, top=0.50in, bottom=0.50in]{geometry}
\author{Daniel Chen}
\date{}
\title{Understanding Social Diffusion Dynamics Using Networked Cognitive Systems}
\begin{document}
    \maketitle
    
    \section*{Abstract}

    Binary decision models with externalities is a class of models that describe
    individuals making a simple binary choice, whose decision depends on other people.
    This class of models is generalizable to infectious diseases, decision making, information spread, fads, riots, etc.
    However, binary decision models are not rich enough to describe the complex process of human attitude formation that eventually lead to behaviors.
    
    The Theory of Reasoned Action is a health behavior model that states behaviors stem from a series of beliefs.
    This theory is used as the computational framework for constraint-satisfaction artificial neural-networks
    which allow us to model psychologically plausible decisions on an individual level.
    This model can be combined with agent-based models, which model health behavior at a population level in order to add psychology into a dynamic social network framework.
    
    This new framework for epidemiological modeling of health behaviors where the agents may exhibit psychological plausible decisions allows us to study the spread of ideas within a social network.
    This will give us insights as to what drives behavioral risk factors for population health behaviors.
\end{document}